\section{Quality}
\subsection{Documentation}
During the overall process, several documents (see Section \ref{sec:deliverables}) will be created in which design choices are made and justified. These documents will be reviewed by both the supervisor and the client, after which the feedback will be processed by the project team. 
\subsection{Version Control}
An existing version control system will be used during all phases of the project to keep track of the history of every created document and source code. One of the benefits of using a version control system is that in case an undesirable change is made to a document, one can revert back to any previous version of the document. The system of choice for this project will be Git\footnote{http://git-scm.com/}, a widely used open-source distributed version control system. It is a system the team is already proficient with and also what the Tribler-team is using at the moment.
\subsection{Code Review}
During the implementation phase, the source code that is written by each project member will be briefly reviewed by the other member. This way, feedback can be obtained that can help improve the overall quality of the source code.\\
A complete source code review will be done by the SIG (see section \ref{sec:sig}), in which the quality of the software is professionally evaluated. By processing their feedback and recommendations, the project members can improve the quality of the software.
\subsection{Software Testing}
During the implementation phase we will use acceptance testing as software testing method, as explained in Section \ref{sec:acc_test}.
\subsection{Evaluation}
After the project, an evaluation will be written by the project team to reflect on how the process went. In this reflection, focus is put on how the overall process, as well as the approach of each team member can be improved. Additional feedback on the bachelor project itself will also be provided. The evaluation does not necessarily increase the quality of the final product itself, but it does increase the quality of future work of the project members, the bachelor project and the final product in the long term by showing what more can be done to expand or improve the final product.


