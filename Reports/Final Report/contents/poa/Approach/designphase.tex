\section{Design Phase}
As a first step in the design phase, functional-, nonfunctional requirements and constraints for the prototype will be elicited. From these requirements, use-cases will be derived that convey how the system should interact with the user. Based on these requirements and use cases, an architectural design document will be created, consisting of a description of the proposed software architecture including subsystem decomposition, persistent data management, global resource handling, concurrency, software control and boundary conditions. Next, a detailed description of the packages, class diagram and a specification of the classes and methods is given in the Technical Design Document. Finally, a Test- and Implementation Plan is created that describes how the different features of the prototype will be tested and implemented.


