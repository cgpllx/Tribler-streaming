\section{VLC}
\thispagestyle{fancy}
The first step into building a streaming application, is checking if VLC can be compiled for Android and if its source code can be modified. Otherwise, a different video decoding framework had to be included in the prototype. VLC was already available for Android, as well as its source code\footnote{source code can be obtained from: git://git.videolan.org/vlc-ports/android.git}. Later on, it is necessary to include VLC in the prototype (see Section \ref{sec:one_apk}). To do this, the source code is needed as well as a method to compile it into an application. VLC was compiled by following the guide in Appendix \ref{sec:app_vlc_build}. This guide was made by collecting information from different resources over the Internet, including the VLC wiki\footnote{https://wiki.videolan.org/AndroidCompile/} and several forums. The guide is also made available on Github (see Section \ref{sec:proj_rep}). Now, VLC has been tested to work on the target device and can be modified to meet the needs of the prototype.