\section{Libtorrent}
\thispagestyle{fancy}
\label{sec:libtorrent_impl}
To build the libtorrent libraries for Android, the Team first tested its functionality on Linux. It was easily compiled to Linux in which the Team ran a client test to see if a torrent file could be downloaded. Android however, has a different compiler, which comes with the Android NDK\footnote{http://developer.android.com/tools/sdk/ndk/index.html}. To compile libtorrent with this compiler was not without errors however. This was due to libtorrent's dependency on Boost.

\subsection{Boost}
This dependency meant that the Boost\footnote{http://www.boost.org/} libraries had to be build for Android as well. In the end, a github repository\footnote{https://github.com/MysticTreeGames/Boost-for-Android} was used which builds Boost for Android after calling a compile script, in which also the version of Boost can be specified.

\subsection{RuTracker}
After the Boost libraries were put in place, the libtorrent library could still not be compiled with the correct compiler. A lot of experimentation was done with different versions of Boost, different compile options and different versions of libtorrent, but without adequate result. Then Egbert Bouman from within the Tribler team came with a tip to look into RuTracker\footnote{http://rutracker.org/forum/index.php}, short for Russian tracker. This application for Android was open source and used libtorrent for its download functionality. After taking out the libtorrrent source and makefiles, the libtorrent libraries could then be build with the ndk-build command. The few errors still remaining were quickly resolved by changing one of libtorrent's configuration options (see appendix \ref{sec:build_libt}). The only drawback of using RuTracker's method is that it can not build the libtorrent libraries for the `mips' and `x86' architectures, meaning that some Intel and Mips tablets are not supported.\\ 
After building the libraries, the Team extracted these and put them in a separate client test project. 

The application could, at this stage, download a torrent and send the downloaded file to the previously build, and separately installed, VLC for playback. The torrent file itself must first be downloaded to the device so it can be selected in the application. At this stage, this works by using a separately installed file-browser, such as ASTRO\footnote{https://play.google.com/store/apps/details?id=com.metago.astro}.