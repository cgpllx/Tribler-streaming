\section{VLC for Android Beta}
VLC for Android Beta is a popular open-source multimedia player and framework\footnote{http://www.videolan.org}, with over 10.000.000 downloads. popular\footnote{https://play.google.com/store/apps/details?id=org.videolan.vlc.betav7neon}. VLC for Android Beta is a ported version of VLC media player for desktop environments. It currently is in a beta phase.
\subsection{Features}
VLC for Android can decode many multimedia file formats\footnote{http://www.videolan.org/vlc/features.html}, using a large number of free decoding libraries, including FFmpeg\footnote{http://www.ffmpeg.org/}. The current version of VLC for Android has support for devices with ARMv6\footnote{http://www.arm.com/products/processors/instruction-set-architectures/index.php}, ARMv7, ARMv7+NEON\footnote{http://www.arm.com/products/processors/technologies/neon.php} and x86 CPU architectures and is written almost entirely in C++. Additional multi-core decoding is supported for Cortex A7\footnote{http://www.arm.com/products/processors/cortex-a/cortex-a7.php}, A9\footnote{http://www.arm.com/products/processors/cortex-a/cortex-a9.php} and A15\footnote{http://www.arm.com/products/processors/cortex-a/cortex-a15.php} processor based devices. Other key features of VLC for Android are automatic screen rotation and touch gesture control.
\subsection{Architecture}
\subsubsection{LibVLCcore}
VLC is a modular framework that consists of one central core: LibVLCcore. This core manages the threads, loading/unloading modules (codecs, multiplexers, demultiplexers, etc.) and all low-level control in VLC.\footnote{http://wiki.videolan.org/Hacker\_Guide/Core/}
\subsubsection{LibVLC}
On top of libVLCcore, a singleton class libVLC acts as a wrapper class, that gives external applications access to all features of the core. Modules on the other hand communicate directly with the core.
\subsubsection{Modules}
VLC comes with more than 200 modules including various decoders and filters for video and audio playback. These modules are loaded at runtime depending on the necessity. Given the modular nature of VLC's architecture, unnecessary modules can be taken out in order to reduce the footprint of the framework.
\subsubsection{Multi-threading}
VLC is a multi-threaded framework. One of the main reasons for using multi-threading is to warrant that an audio or video frame will be played at the exact presentation time without blocking the decoder threads.
\subsection{Documentation}
Ample documentation is available that covers both the core as well as the modules. Additional Android-specific documentation regarding the source code compilation and debugging process is also available\footnote{https://wiki.videolan.org/AndroidCompile/}. An extensive knowledge base is also publicly available, in which codecs, file formats and protocols are documented.\footnote{https://wiki.videolan.org/Knowledge\_Base/}.
\subsection{Updates}
The VLC for Android application is backed up by a non-profit organization called VideoLAN\footnote{http://www.videolan.org/videolan/} and at the time of writing this report, the source code is updated on almost a daily basis.\footnote{http://git.videolan.org/?p=vlc-ports/android.git;a=summary}
\subsection{Performance}
\label{sec:VLC_performance}
A small performance test of VLC is done by playing two videos encoded in h.264, one with a resolution of 720p, and the other with 1080p. These are popular video formats used in encoding videos. This test is done on the Google Nexus 10" tablet device, which has VLC installed from the Google Play Store. The result is that VLC for Android features smooth HD playback, and no frame lag has been detected. The automatic screen rotation is also fast and smooth. Touch gesture control also feels responsive and looks smooth. 


