\section{Risk Factors}
\subsection{VLC for Android incompatible with target device}
Given the fact that VLC for Android is currently still in a beta phase, there is a chance that the framework will be incompatible with the target device of the project: the Google Nexus 10"\footnote{http://www.google.com/nexus/10/}.\\
\newline
\textit{Indications:}
\begin{itemize}
	\item[-]Component fails to compile for target device architecture.
	\item[-]Component fails to install on target device.
	\item[-]Component fails to run on target device.
	\item[-]Component crashes unexpectedly during runtime.
	\\
\end{itemize}
\textit{Countermeasure:}
\newline
The VLC for Android component will be dropped and the native Android multimedia framework (Stagefright) will be used.\\
\newline
\textit{Timing of countermeasure action:}\\
When the above listed indications appear during initial testing of the component on the target device, the countermeasure will take place. Initial testing of this component will be performed at the end of the Orientation phase.\\
\newline
\textit{Impact:}\\
As a result of dropping the VLC for Android component, the number of supported multimedia formats in the prototype will be significantly reduced. However, the fewer codecs that are supplied with Stagefright are hardware accelerated and are guaranteed to work, as these hardware-based codecs are supplied by the hardware vendor. The `must have' functionalities of the prototype will therefore not be compromised by this solution.\\
\subsection{VLC for Android not working in combination with Tribler components}
Given the bleeding edge nature of the combination of VLC for Android with P2P streaming data, there exists a risk that this video decoding framework will not work with key components of Tribler.\\
\newline
\textit{Indications:}
\begin{itemize}
	\item[-]Component fails to play video data from Tribler components.
\end{itemize}
\textit{Countermeasure:}\\
The VLC for Android component will be dropped and the native Android multimedia framework (Stagefright) will be used.\\
\newline
\textit{Timing of countermeasure action:}\\
When there is no concrete evidence of a working combination of VLC for Android and the key components of tribler before the end of the first week of the implementation phase.\\
\newline
\textit{Impact:}\\
As a result of dropping the VLC for Android component, the number of supported multimedia formats in the prototype will be significantly reduced. However, the fewer codecs that are supplied in combination with Stagefright are hardware accelerated and are guaranteed to work, as these hardware-based codecs are supplied by the hardware vendor. The 'must have' functionalities of the prototype will therefore not be compromised by this solution.\\ 
\subsubsection{Python for Android}
In case Python for Android does not work on the target devices, the Tribler core will not be able to be ported to Android, since this is completely written in Python.\\
\newline
\textit{Countermeasure:}\\
A plug-in for VLC will be written that will make use of the libswift\footnote{http://libswift.org/} library.\\
\newline
\textit{Impact:}\\
Many of the unique Tribler functionalities will not be available, and the prototype application will not be easily extendable towards having Tribler functionalities. The BitTorrent engine which Tribler relies on: libtorrent, will be implemented in this case.
\subsection{Frame rate performance issue}
\subsubsection{VLC}
Given the fact that VLC for Android is currently still in a beta phase, the playback performance with some multimedia file formats can be unacceptable\\
\newline
\textit{Countermeasure:}\\
Based on how many file formats are affected by the performance issue, the VLC component will be dropped and the native Android multimedia framework (Stagefright) will be used, or individual modules responsible for decoding will be adapted, if possible, or the formats will be reported as not functioning correctly and will not be supported in the prototype application.\\
\newline
\textit{Impact:}\\
In the worst case, the number of supported multimedia codecs in the prototype will be significantly reduced. However, the fewer codecs that are supplied in combination with Stagefright will be hardware accelerated and are guaranteed to work, as these hardware-based codecs are supplied by the hardware vendor. The 'must have' functionalities of the prototype will therefore not be compromised by this solution.\\
\newline
In the best case, only a few uncommon multimedia file formats will be unsupported in the prototype application, which will hardly affect its video decoding capabilities.





