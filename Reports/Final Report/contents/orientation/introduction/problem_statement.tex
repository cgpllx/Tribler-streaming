\section{Problem Statement}
The goal of our research is to come towards such a solution. The following research question is therefore central for this research:\\
\\
\textit{``How can we make video-on-demand available for mobile devices using a non-centralized approach?''}\\

This research question consists out of three main elements: VoD, mobile devices and a non-centralized approach.

VoD is the main feature that needs to be considered for implementation. Ideally, we can use an existing tool or library for VoD on mobile devices. In the first part of our research, we will focus on the following research question:

\newcounter{qcounter}
\begin{list}{Research Question \arabic{qcounter}:~}{\usecounter{qcounter}}

	\item\textit{``Which solutions exist for VoD on (non-)mobile devices''}\\
	
	If no solutions exist for mobile devices, there may be solutions for non-mobile devices which can be ported to those mobile devices. This leads to the following sub question:\\

		\begin{enumerate}[(a)]
			\item\textit{``How can a port of an existing non-mobile solution be made to mobile devices?''}\\
			
			Important to realize is, that a port of a solution might not have full functionality because of hardware issues or other incompatibilities. Therefore research has to be done into the following:\\
			
			\item\textit{``What are the limitations of such a ported solution to mobile devices?''}\\
		\end{enumerate}

	Interesting to see is which of these solutions, use a non-centralized approach. Non-centralization is key to getting a low-cost, highly reliable and community driven solution (see Section: \ref{sec:central}). Therefore, in the second part of our research, the focus will be put on: \\

	\item\textit{``Which solutions for VoD use a non-centralized approach?''}\\

	There are an abundant number of different mobile devices, ranging from different sizes and processors to different platforms. Each different platform, such Apple's iOS, Android and Windows phone, has its advantages and disadvantages. These pros and cons have to be researched in order to see which platforms are best suitable for which solutions, hence, the following research question:\\

	\item\textit{``Which mobile platforms are best suited to implement VoD in a non-centralized approach?''}\\
	
	Videos are encoded to save space and bandwidth once transmitted. To decode these videos, decoding software has to be used. This software depends on the type of hardware it runs on and has to be included or linked to in order to play the video, research into this matter will be led by the following sub question:\\

	\item\textit{``Which solutions exist for video decoding on the chosen platform?"}\\
\end{list}

The different mobile platforms are described in Chapter \ref{sec:mos}, covering Research Question 3. Research Question 1, about (non-)mobile VOD solutions, will be discussed in Chapter \ref{sec:vod}. In this same chapter, the non-centralization aspect (Research Question 2) as well as how a port can be made to mobile devices (Research Question 1(a)) are described. The limitations of these ports are also described in this section (Research Question 1(b)). Video decoding is described in Chapter \ref{sec:videc}. Finally an analysis of the risks involved is included in Chapter \ref{sec:ra}, which describes what to do when a situation arises that compromises a solution or part of it.