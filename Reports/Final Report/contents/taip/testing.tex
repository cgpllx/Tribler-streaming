\chapter{Testing}
\thispagestyle{fancy}
\label{sec:test}
This chapter provides insight into how testing will be executed in this project. The first section will explain how unit testing will be incorporated  in the project, which is followed by a section about acceptance testing.

\section{Unit Testing}
Every function or method can be modeled as a black box; the black box processes the input and generates the output. To test if the correct output is generated, unit testing is used. Within a unit test the tester can define the input and what the expected output is, then the unit test runs to test if the function's output is the same as the expected output. If not, it will warn the tester that the test failed so the function can be changed to behave correctly. 
In this project the team will test all methods that have adequate logic in them. For example, getters and setters will not be tested. Methods from third-party software like Tribler, libtorrent and VLC will also not be tested separately. 

\section{Acceptance testing}
Acceptance testing means that the team will see if the software in the current state meets the requirements by manually testing the functionality. In the different sprints, functionality will be added to the prototype. Acceptance testing will be done in between the sprints to ensure that the prototype meets the requirements set to be implemented for that sprint.