\part{Introduction}
The Parallel and Distributed Systems group (hereafter: PDS) is a research group within the Software and Computer Technology (hereafter: SCT) department, which is part of the Faculty Electrical Engineering, Mathematics and Computer Science (hereafter: EEMCS) of the Delft University of Technology. The research within the PDS group concentrates on the modeling, the design, the implementation, and the analysis of parallel and distributed systems and algorithms. Most of this research is experimental: the aim is to build prototypes of systems, preferably used in the real world, to demonstrate the quality of the proposed solutions. The main research areas of the PDS group are Peer-to-Peer (hereafter: P2P) systems and online social networks, massive multi-player online games, grids, clouds, multi-core architectures and parallel programming. 
P2P is considered by many as an efficient, reliable, and low cost mechanism for distributing any media file or live stream, and it is used extensively \cite{internet_study}. Much of the current research activities in P2P within the PDS group are centered around Tribler\footnote{http://www.tribler.org}. Tribler is an application that enables its users to find, consume and share content through a P2P network. Tribler builds on BitTorrent\footnote{http://www.bittorrent.com} and is available on desktop environments. Currently mobile internet traffic continues to consistently gain on desktop traffic in terms of volume\footnote{http://gs.statcounter.com/mobile\_vs\_desktop\-ww\-monthly\-200812\-201306} and mobile traffic is estimated to surpass traffic from wired devices in 2017\footnote{\url{http://www.cisco.com/en/US/solutions/collateral/ns341/ns525/ns537/ns705/ns827/VNI\_Hyperconnectivity\_WP.pdf}}. In response to this growth and to meet the increasing demands of the market, the development of a mobile version of Tribler would be of great value. \\

First, the plan of action is described in which the project background, assignment, approach and design are described, as well as how the quality of the product will pursued. This is followed by the orientation phase, in which different tools, methodologies and risks are researched. Part \ref{sec:req_part} describes the requirements which were set up together with the Client. The next part is about the test and implementation plan, which was set up to guide the developers towards a working prototype. How the system is designed beforehand is explained in Part \ref{sec:arch_design_part}. Part \ref{sec:impl_part} is a chronological report about how the process of the implementation phase developed. The final part includes the conclusion, what future work lies ahead and a personal reflection.
