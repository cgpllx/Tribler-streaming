\section{VLC for Android Beta}
VLC for Android Beta is an open-source multimedia player and framework\footnote{http://www.videolan.org}, that is immensely popular and has been downloaded over 10 million times.\footnote{https://play.google.com/store/apps/details?id=org.videolan.vlc.betav7neon\&hl=nl} VLC for Android Beta is a ported version of VLC media player\footnote{http://www.videolan.org/vlc/}. It currently is in a beta phase, meaning that it is not completely finished, lacks features, and can still contain unknown issues.
\subsection{Features}
VLC for Android can decode many multimedia file formats, using a large number of free decoding libraries, including FFmpeg. The current version of VLC for Android has support for devices with ARMv6\footnote{http://www.arm.com/products/processors/instruction-set-architectures/index.php}, ARMv7, ARMv7+NEON\footnote{http://www.arm.com/products/processors/technologies/neon.php} and x86 CPU architectures and is written almost entirely in C++. Additional multi-core decoding is supported for Cortex A7\footnote{http://www.arm.com/products/processors/cortex-a/cortex-a7.php}, A9\footnote{http://www.arm.com/products/processors/cortex-a/cortex-a9.php} and A15\footnote{http://www.arm.com/products/processors/cortex-a/cortex-a15.php} processor based devices. Other key features of VLC for Android are automatic screen rotation and touch gesture control.
\subsection{Architecture}
\subsubsection{Core}
VLC is a highly complex modular framework that consists of one central core: LibVLCcore. This core manages the threads, loading/unloading modules (codecs, multiplexers, demultiplexers, etc.) and all low-level control in VLC.\footnote{http://wiki.videolan.org/Hacker\_Guide/Core/}
\subsubsection{LibVLC}
On top of libVLCcore, a singleton class libVLC acts as a wrapper class, that gives external applications access to all features of the core. Modules on the other hand communicate directly with the core.
\subsubsection{Modules}
VLC comes with more than 200 modules including various multiplexers, demultiplexers, decoders and filters. These modules are loaded at runtime depending on the necessity. Given the modular nature of VLC's architecture, unnecessary modules can be taken out in order to reduce the footprint of the framework.
\subsubsection{Multi-threading}
VLC is a multi-threaded framework. One of the main reasons for using multi-threading is to warrent that an audio or video frame will be played at the exact presentation time without blocking the decoder threads.
\subsection{Documentation}
Ample documentation is available that covers both the core as well as the modules. Additional Android-specific documentation regarding the source code compilation and debugging process is also available\footnote{https://wiki.videolan.org/AndroidCompile/}. An extensive knowledge base is also publicly available, in which codecs, file formats and protocols are documented.\footnote{https://wiki.videolan.org/Knowledge\_Base/}.
\subsection{Updates}
The VLC for Android application is backed up by a non-profit organization called VideoLAN\footnote{http://www.videolan.org/videolan/} and at the time of writing this report, the source code is updated on almost a daily basis.\footnote{http://git.videolan.org/?p=vlc-ports/android.git;a=summary}
\subsection{Performance}
The current version of VLC for Android features very smooth HD playback, and no frame lag has been detected. The automatic screen rotation is also very fast and smooth. Touch gesture control is also working fine.


