\section{Trade-off}
In this section the different criteria for what makes up a good VoD service are elucidated. Also described in this section is in how the different services perform on these criteria.

\subsection{Non-centralization of servers}
\label{sec:central}
When no central servers are used, the service has low hardware maintenance costs. Further more in a central server way, if the server fails, data might be lost and the service is out of order, this can not happen in a non-centralized approach with enough users, making the non-centralized approach highly reliable. This also means that every user is part of the solution, making it a community driven approach, where every user has influence on the system(in terms of availability and range of content), the size of this influence depends on the number of users connected.
Of the VoD services that were previously described, only Tribler currently uses a non-centralized approach. To implement non-centralization into one of the other services would be far beyond the scope of this project due to the size and state of maturity of the those services and therefor the time it would take to switch to non-centralization. 
\subsection{Service costs}
While Youtube, ITV Player and Tribler are free of cost, Netflix and Amazon Instant Video are not. Youtube does show advertisement in some of its videos, as does ITV player. If the services have a different price, the quality of service matters, this can be measured by examining how the services perform on the following criteria:
\begin{itemize}
\item The amount of videos available\\
Of the different services, Youtube has the most available content. Followed by Amazon Instant Video which boasts a large array of movies and shows due their success with the Amazon web store. Tribler has a lot of different content but relies on the different peers for availability of the content in the swarm. The other services have a lower amount of available videos.
\item The sort of videos available\\
While Youtube shows TV Programs, movies and user uploaded content, most currently airing TV shows and movies are copyrighted and will therefor not be available on Youtube. ITV Player, Netflix and Amazon Instant Video do provide these TV programs and movies, but do not provide a way of showing user created content. Tribler includes all three categories, although not many users upload their created content.
\item The quality of the videos available\\
For the video quality Youtube and Tribler depend on which quality the user chooses to upload, but a quick browse through a number of different videos show that most new content is available in  High Definition(HD, 720p and up\footnote{http://hometheater.about.com/od/hometheaterglossary/g/720pdef.htm}). Amazon Instant Video charges extra for HD content. Netflix and ITV Player lets the user choose between HD and lower qualities.
\item Supported devices\\
The different devices on which the services run are listed in section \ref{sec:vodserv}. Currently, Tribler runs on desktop environments.
\end{itemize}
\subsection{Portability}
As said earlier, Android is the platform of choice to provide the VoD functionality. Tribler and Amazon Instant Video are not yet available for this platform and must be ported to Android to make VoD possible. Amazon Instant Video does have an application on Android\footnote{https://play.google.com/store/apps/details?id=com.amazon.avod}, but it works only on select devices and the ratings are quite low due to incompatibility. Since Amazon Instant Video is a closed-source project, porting it is nearly impossible. Tribler is open-source, but written in Python. There is a project on GitHub\footnote{https://github.com/d3vgru/python-for-android/tree/tgs-android} that provides the functionality of running Python code on Android, so this is not a significant impediment. The limitation of this port could however be that it runs slower than making Tribler for Android in Java. Also, how video playback works on desktop environments is different than on Android, section \ref{sec:videc} will go into more detail about this.
\section{Conclusion}
In conclusion, Tribler is chosen as the VoD service to implement on Android. Although it is not yet available for Android, it can be ported and it uses a non-centralized approach, is free of costs for the user, has a large amount of videos available in HD and shows no advertisements.