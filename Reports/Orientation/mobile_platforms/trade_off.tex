\section{Trade-off}
The trade-off will be made according to several criteria. First the market share of the different mobile platforms will be described, followed by the proficiency of the team in relation to the different programming languages. 

\subsection{Market share}
In the second quarter of 2013, research was conducted by Strategy Analytics\footnote{http://www.strategyanalytics.com/} in to the global market share of the different mobile platforms. The result was that Android holds a 79.5\% market share, iOS has a 13.6\% share and Microsoft has a share of 3.9\%\footnote{http://news.cnet.com/8301-1035\_3-57596548-94/android-nabs-record-80-percent-market-share-in-q2/}.  By far, Android has the biggest share and this means that an application in Android will have a larger user base than its iOS or Windows Phone counterpart, which is positive for the Client

\subsection{Programming language}
The team is proficient with C, C++, C\#, Java and script languages such as Javascript, XML, PHP, etc. They also have some experience with making applications for Android. Further more they do not own a Mac computer or iPhone. 

\subsection{Conclusion}
Concluding, Android is chosen as mobile platform because it has the largest market share and the team is more capable of developing applications for Android than for any other mobile platform.